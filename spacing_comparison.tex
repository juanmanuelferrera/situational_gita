\documentclass[12pt,twoside]{book}
\usepackage{fontspec}
\setmainfont{Libertinus Serif}
\usepackage[paperwidth=6in,paperheight=9in]{geometry}
\geometry{
  inner=19mm,
  outer=14mm,
  top=13mm,
  bottom=13mm,
  headheight=12pt,
  headsep=7mm,
  footskip=10mm
}
\usepackage{setspace}
\setstretch{1.15}

\begin{document}

% TEST WITH INDENTATION (CURRENT STYLE)
\chapter{Chapter Test - Indented Style}
\setlength{\parindent}{1.5em}
\setlength{\parskip}{0pt}

\section*{Opening Story}
Marcus hadn't slept well in three weeks. The merger announcement had come down like a hammer, and now he sat in his corner office at 6:47 AM, staring at an email that made his jaw clench.

Derek—his colleague, his supposed friend—had sent it to the entire executive team. Just a simple suggestion, really. A recommendation that Marcus's division be "restructured for efficiency." Corporate-speak for dismantled.

The anger wasn't immediate. It built slowly, like water heating on a stove. First the tightness in his chest. Then the heat rising to his face. Then the familiar sensation of his thoughts accelerating, spinning, searching for the perfect response that would destroy Derek's credibility forever.

\section*{Understanding}
We've all been there. Maybe not throwing coffee mugs, but we've all felt that moment when anger stops being an emotion and becomes a force. When it takes over. When we become the anger rather than the person experiencing it.

This is where most advice on anger fails. It tells you to breathe deeply, to count to ten, to imagine a peaceful place. As if anger were just a mood that could be shifted with the right mental trick.

But the Bhagavad-gītā doesn't treat anger like a mood. It treats it like a mechanism. A precise, predictable sequence of events that leads to destruction. And once you understand the mechanism, you can interrupt it.

\cleardoublepage

% TEST WITH VERTICAL SPACING
\chapter{Chapter Test - Vertical Space Style}
\setlength{\parindent}{0pt}
\setlength{\parskip}{4pt plus 0.5pt minus 0.5pt}

\section*{Opening Story}
Marcus hadn't slept well in three weeks. The merger announcement had come down like a hammer, and now he sat in his corner office at 6:47 AM, staring at an email that made his jaw clench.

Derek—his colleague, his supposed friend—had sent it to the entire executive team. Just a simple suggestion, really. A recommendation that Marcus's division be "restructured for efficiency." Corporate-speak for dismantled.

The anger wasn't immediate. It built slowly, like water heating on a stove. First the tightness in his chest. Then the heat rising to his face. Then the familiar sensation of his thoughts accelerating, spinning, searching for the perfect response that would destroy Derek's credibility forever.

\section*{Understanding}
We've all been there. Maybe not throwing coffee mugs, but we've all felt that moment when anger stops being an emotion and becomes a force. When it takes over. When we become the anger rather than the person experiencing it.

This is where most advice on anger fails. It tells you to breathe deeply, to count to ten, to imagine a peaceful place. As if anger were just a mood that could be shifted with the right mental trick.

But the Bhagavad-gītā doesn't treat anger like a mood. It treats it like a mechanism. A precise, predictable sequence of events that leads to destruction. And once you understand the mechanism, you can interrupt it.

\end{document}
